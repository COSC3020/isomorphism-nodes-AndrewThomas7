%This is a Tex Version of Math 304 Homework Template
%
\documentclass{amsart}
\setlength{\textheight}{9in}
\setlength{\topmargin}{-0.5in}
\setlength{\textwidth}{7in}
\setlength{\evensidemargin}{-0.25in}
\setlength{\oddsidemargin}{-0.25in}
\usepackage{amsfonts}
\usepackage[utf8]{inputenc}
\usepackage[T1]{fontenc}
\usepackage{graphicx}    
\usepackage{parskip}
\usepackage{amsmath, amssymb}% needed to include these graphics
%\graphicspath{{./Pictures/}}      % only in case you want to keep the pictures in a separate
                                  % subdirectory; also see the appropriate line below
\usepackage{caption}
\usepackage{subcaption}
\usepackage{float}
\newcounter{temp}
\theoremstyle{definition}
\newtheorem{Thm}{Theorem}
\newtheorem{Prob}{Problem}
\newtheorem*{Def}{Definition}
\newtheorem*{Ans}{Answer}
\newcommand{\dis}{\displaystyle}
\newcommand{\dlim}{\dis\lim}
\newcommand{\dsum}{\dis\sum}
\newcommand{\dint}{\dis\int}
\newcommand{\ddint}{\dint\!\!\dint}
\newcommand{\dddint}{\dint\!\!\dint\!\!\dint}
\newcommand{\dt}{\text{d}t}
\newcommand{\dA}{\text{d}A}
\newcommand{\dV}{\text{d}V}
\newcommand{\dx}{\text{d}x}
\newcommand{\dy}{\text{d}y}
\newcommand{\dz}{\text{d}z}
\newcommand{\dw}{\text{d}w}
\newcommand{\du}{\text{d}u}
\newcommand{\dv}{\text{d}v}
\newcommand{\ds}{\text{d}s}
\newcommand{\dr}{\text{d}r}
\newcommand{\dth}{\text{d}\theta}
\newcommand{\bbR}{\mathbb{R}}
\newcommand{\bbN}{\mathbb{N}}
\newcommand{\bbQ}{\mathbb{Q}}
\newcommand{\bbZ}{\mathbb{Z}}
\newcommand{\bbC}{\mathbb{C}}
\newcommand{\dd}[2]{\dfrac{\text{d}#1}{\text{d}#2}}
\newcommand{\dydx}{\dfrac{\text{d}y}{\text{d}x}}
\renewcommand{\labelenumi}{{\normalfont \arabic{enumi}.}}
\renewcommand{\labelenumii}{{\normalfont \alph{enumii}.}}
\renewcommand{\labelenumiii}{{\normalfont \roman{enumiii}.}}
\font \bggbf cmbx18 scaled \magstep2
\font \bgbf cmbx10 scaled \magstep2
\begin{document}
\begin{center}
	{\bgbf COSC 3020\\ \smallskip
	Algorithms\\ \smallskip
	Spring 2025\\ \smallskip
	Isomorphism-Nodes\\ \smallskip
        Andrew Thomas}
\end{center}
\smallskip

\begin{enumerate}
	\item Prove that if two graphs 
A
 and 
B
 do not have the same number of nodes, they cannot be isomorphic.

\section*{Answer}

\subsection*{Proof}

Assume for the sake of contradiction that two graphs 
\[ A=(V_1,E_1), \quad B=(V_2,E_2) \]
are isomorphic, such that there exists some function 
\[
\phi: V_1 \longrightarrow V_2
\]
that is one-to-one and onto, and 
\[
(u,v) \in E_1 \iff (\phi(u), \phi(v)) \in E_2,
\]
and that \(|V_1| \ne |V_2|\).

\subsubsection*{Case 1: \(|V_1| > |V_2|\)}

We now construct the following map:
\[
\phi:
\begin{pmatrix}
v_{1,1} & v_{1,2} & \cdots & v_{1,k} & \cdots & v_{1,n} \\
v_{2,1} & v_{2,2} & \cdots & v_{2,k}
\end{pmatrix}
\]
where \(n, k \in \mathbb{N},\ n > k\), and \(v_{1,i} \in V_1\), \(v_{2,i} \in V_2\).

We now use the definition of an injective (one-to-one) function to derive a contradiction.

\textbf{Definition:} A function \(\phi: V_1 \rightarrow V_2\) is said to be injective if for all \(v_1, v_2 \in V_1\),
\[
\phi(v_1) = \phi(v_2) \Rightarrow v_1 = v_2.
\]

To show that this doesn't work in this case, we want to find \(v_1 \ne v_2\) such that \(\phi(v_1) = \phi(v_2)\).\par

Take $v_{1,k'},v_{1,k'+1}\in V_1$ such that $k<k'<n$. Now note that both  $v_{1,k'},v_{1,k'+1}$ are different vertices so $v_{1,k'}\neq v_{1,k'+1}$ however when we map them it must be the case that we have $\phi(v_{1,k'})=\phi(v_{1,k'+1})$ for some $v_{1,k'},v_{1,k'+1}\in V_1$ since $k<k'$ and elements in $V_2$ only index up to $k$.

In lamens terms this must be the case because there are more nodes in $V_1$ than there are in $V_2$, meaning at some point in the mapping more than one node from $V_1$ will get mapped to the same element in $V_2$.\par
This is a contradiction because we assumed the map was bijective.
\subsubsection*{Case 2: \(|V_1| < |V_2|\)}

We now construct the following mapping:
\[
\phi:
\begin{pmatrix}
v_{1,1} & v_{1,2} & \cdots & v_{1,k} & & & \\
v_{2,1} & v_{2,2} & \cdots & v_{2,k} & \cdots & v_{2,n}
\end{pmatrix}
\]

\textbf{Definition:} A function \(\phi: V_1 \rightarrow V_2\) is not surjective if there exists some \(u \in V_2\) such that for all \(v \in V_1\), \(\phi(v) \ne u\).

Choose $u=v_{2,k'}$ where $k'\in \mathbb{N}$ and $k<k'<n$ then because $k'>k$ $\forall$ $v_{1,k}\in V_1$ it must be the case that $\phi(v_{1,k})\neq v_{2,k'}$

In other words because there are more noddes in $V_2$ than $V_1$ there are simply some values you cannot map to at all. Thus we can see that this mapping cannot meet the criteria for a surjective function..

\bigskip

\noindent
\textbf{Conclusion:} Therefore it cannot be the case that $A\cong B$  if \(|V_1| \ne |V_2|\).
\end{enumerate}
\end{document}
